\input{../header.tex}
\usepackage{hyperref}
\title{MGT 646: Start-up Founder Practicum}
\author{Spring 2015; 2 or 4 Units; L400 Evans Hall}
\date{}
                                       % Activate to display a given date or no date

\begin{document}
\maketitle

\begin{center}
\begin{tabularx}{\textwidth}{LL}\toprule
Kyle Jensen (Instructor) & Hanna German (Senior Admin) \\

\textbf{Email:} kyle.jensen@yale.edu &
\textbf{Email:} hanna.german@yale.edu\\

\textbf{Office phone:} 203-436-9650 &
\textbf{Office phone:} 203-436-5039\\ \midrule

Jennifer McFadden (Mentor) &
Robert Bettigole (Mentor) \\

\textbf{Email:} jennifer.h.mcfadden@gmail.com &
\textbf{Email:} rob@elmvc.com\\

\textbf{Office phone:} 203-436-5039 &
\textbf{Office phone:} (Contact Hanna) 
\\

\bottomrule

\end{tabularx}
\end{center}

\subsection*{Overview}
The purpose of this course is to provide full-time SOM students with a mechanism to work on their start-up ventures for credit, applying principles derived from their other coursework, particularly the integrated core curriculum. Students in this course articulate milestones for their ventures and work with faculty, staff, and mentors to meet those milestones. Generally, the course employs ``lean'' methodology. Admission to the course is restricted to students in a full time program at Yale SOM who have formed a venture prior to the beginning of the class. Not all team members must take the class. Admitted students are given working space in the Entrepreneurial Studies Suite of Yale's Evans Hall.

Admission is by application only and is limited to SOM students. To apply for Spring 2015, please fill out the application at \url{http://goo.gl/Fq2X60} and provide all relevant supplementary
 documents. Applications are due by 11:59 PM on 10/31. Applicants submitting the most compelling ideas for ventures will be contacted by 11/3 to set up a time for a brief interview. If you have any questions, please contact Jennifer McFadden, Associate Director
 of Entrepreneurial Programs, at jennifer.mcfadden@yale.edu. 

\subsection*{Objectives}
This course is a replacement for independent study, which was previously the only mechanism
whereby SOM students could work on their start-up companies for credit. The course has the
following objectives.
\begin{enumerate}
	\item
		It will make your entrepreneurial endeavor part of your
		educational experience here at Yale SOM.
	\item
		It will put you in close contact with the entrepreneurship
		faculty and other founders via regular meetings and your
		co-location in the Entrepreneurial Studies Suite, L400 Evans Hall.
	\item
		It will help your new venture succeed.
\end{enumerate}

\subsection*{Immediate action items}
You must visit Hanna German in L400 as soon as possible.

\begin{itemize}
	
		\item
		Hanna will help you choose a desk in L400. There are not enough
		desks. If you are content sharing one of the two flex spaces,
		please tell Hanna. When you have a desk, please label it
		clearly. (Feel free to decorate it.) You are the only students
		in Evans Hall with their own desks. These desks are first come, first
		served. If you do not use the desk, the faculty will ask you to give
		the space to one of the many entrepreneurs who are not enrolled in the
		practicum that require work space.
	
	\item Hanna will invite you to the Slack.com account for the Program on Entrepreneurship.
		This will let you 
		chat with faculty and other founders. You can download the Slack app
		on all platforms: Windows, Mac, Android, iOS, etc.
		
	\item Coordinate with Hanna, Jennifer & Rob Bettigole to find a time that works for your weekly check in.
\end{itemize}


\subsection*{Structure and grading}

The course works roughly as follows.
\begin{itemize}
	\item You will meet with Kyle Jensen and Jennifer McFadden\footnote{Jennifer McFadden is the Associate Director of Entrepreneurial Programs at the Yale School of Management.}
		initially to discuss
		the state
		of your venture, its goals, and your personal goals as a founder.
		Together, you will decide upon milestones for the semester. These
		milestone may be specific to the venture, but some may also be related
		to leadership development and similar personal ends. The nature of your
		milestones will depend on the age of your venture, the market,
		the number of co-founders, your funding, and whether you are taking
		the course for two or four units. It is something upon
		which we will agree by mutual consent.
	\item You will meet every week with the instructor or one of the
			mentors for 30 minutes. You will need to bring a ``dashboard''
			with you to these meetings. The dashboard should be no longer than a single page and include the following items, in bulleted form:
			\begin{itemize}
				\item A summary of what you accomplished over the past week.
				\item A list of your key tasks for the week ahead.
				\item A brief discussion of any specific issues you are encountering.
				\item How the instructor and
					mentors may help you, now.
			\end{itemize}
	\item We encourage you to develop your own internal tools for communicating with your team and allocating tasks. Great project management tools include Basecamp, Asana, and Evernote. We are happy to work with you at the beginning of the semester to develop a project management process that meets the needs of your team and to create a reporting mechanism that integrates with this process.
	\item You will be graded based on the progress you make towards the
		milestones you articulated at the beginning of the semester.
		To help you understand your progress, after each meeting the
		instructor or mentor with whom you meet will leave you with an
		informal evaluation of your progress: exceptional, on-track, or
		off-track. These interim evaluations will inform, but not
		determine, your final 
		grade for the course.

\end{itemize}

\subsection*{Additional Course Requirements}

This course is an opportunity for you to interact with other entrepreneurs from the School of Management and to build a network that you can leverage as you move forward with your venture and your career. As such, we encourage you to work out of L400 and to participate in relevant entrepreneurial events around Yale SOM. In addition to the above course requirements, students in MGT 646 are required to:

\begin{itemize}
	\item Attend a weekly "All Hands" meeting in L400 (time TBD, will include food). This meeting provides an opportunity for you to get feedback on your venture and to gain experience pitching your idea to a group of people. The purpose of the meeting is to provide a forum where you may:
		\begin{itemize}
				\item Pitch your venture.
				\item Discuss a particular problem that you're encountering.
				\item Get feedback on product or branding questions.
				\item Have fellow students test your product.
				\item Discuss your overall strategy and solicit advice.
			\end{itemize}
	\item In addition to the weekly "All Hands," we require that you draft one blog post per team per month detailing your venture and the progress that you are making. These blog posts will be published on the Yale SOM Entrepreneurial Programs blog, as well as on your own site. The goal is to help you become more comfortable crafting stories around your idea and venture. The mentors will be available to provide feedback on these posts.
	\item We have collected a set of resources that you can access as you move forward with your venture. Although we do not require that you read the full set of articles/books provided, we may occasionally ask you and your team to go over material in a particular section and to work on the associated "To Do's." This is for your benefit. If we ask you to read the materials, please make sure that you do so.

\end{itemize}

\subsection*{Important notes}

This is an unusual course. As such, there are some additional components
we are including here.

\begin{itemize}
	\item For your weekly meetings, you will be paired with one of the mentors, or a mentor not listed above
		that is pre-approved by the instructor.
	\item The instructor, Kyle Jensen, will join all meetings as he is able.
	\item If one of the mentors is over-burdened or unavailable, we will
		ask you to meet with a mentor that has more availability.
	\item You can meet with different instructors or mentors as
		you like, depending upon their availability. However, for the sake of continuity, we expect that you will meet with the same mentor every week for your weekly check-ins. This will help us better track your progress and gauge areas where you may need help.
			\item Co-founders taking the practicum together should meet simultaneously.
	\item You may book meetings with Kyle Jensen and Jennifer McFadden
		by contacting Hanna German in person in L400 Evans Hall. Rob
		Bettigole will likely have recurring, consistent hours of availability.
		These will likely be reserved using a sign-up sheet in L400.
	\item The faculty and staff worked hard to obtain the office space
		in L400 for you. \emph{Please use it and, if you do not need
		it, please let the instructor know as soon as possible.}
		There are many entrepreneurs at SOM who need working space.
	\item Please be respectful of the founder office space in L400 and
		respectful of your peers occupying that space. That caveat aside,
		you are free to use L400 at all hours.
	\item If you have questions, needs, or concerns
		related to L400, please speak with Hanna German.
	\item Use the Slack account for communicating with the residents of
		L400. There is an \#l400 channel that is perfect for asking people
		to wash their dishes, etc. (Of course, that is not a realistic
		example because nobody will need to be
		asked to wash their dishes.)
	\item Rob Bettigole is an Executive Fellow at SOM in the Spring of 2015
		and a mentor for MGT 646.
		He is also a venture capitalist, founder of Elm Street Ventures,
		alum of SOM, and generously donating his time to us in the support
		of Yale SOM entrepreneurs.
		is a non-trivial possibility Rob would, at some point, consider an
		investment in your company. Please treat Rob as a potential
		investor first and a mentor for MGT 646 second.  
	\item The faculty and staff of the Program on Entrepreneurship---Kyle Jensen,
		Jennifer McFadden, and Hanna
		German---have a pedagogical relationship with you. They will 
		use their best judgment to determine what information about
		your venture should be held in confidence. Further, they will
		use best efforts to recuse themselves from decisions that could
		be influenced by confidential information about your venture.
		For example, these persons will not make decisions about scholarships
		awarded by the program. There is one important exception to this
		rule: the faculty and staff will feel free to comment about your
		character and drive as an entrepreneur. It is incumbent upon you
		to demonstrate those attributes.
\end{itemize}



\end{document}  